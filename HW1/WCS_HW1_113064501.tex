%\documentclass[A4,12pt]{article}
\documentclass[letterpaper,12pt]{article}
\usepackage{epsfig}
\usepackage{float}
\usepackage{amssymb,amsmath,latexsym}
\usepackage[labelformat=empty]{caption}
\usepackage{color}
%\usepackage[abs]{overpic}

\usepackage{graphicx}
\usepackage{epstopdf}

\usepackage{tikz}
\usetikzlibrary{arrows}
\usetikzlibrary{calc}
\usetikzlibrary{scopes}
\usetikzlibrary{shadows}
\usetikzlibrary{chains}
%\usetikzlibrary{shadows.blur}

\topmargin      0in 
\textheight     9.0in 
\headheight     -0.0in 
\headsep        0in
\textwidth      6.5in 
\oddsidemargin  0in 
\evensidemargin 0in
\parskip        0pt

\newcommand{\slfrac}[2]{\left.#1\middle/#2\right.}
\newcommand{\bm}    [1]{\mbox{\boldmath $#1$}}

\newtheorem{thm}           {Theorem}
\newtheorem{lemma}    [thm]{Lemma}
\newtheorem{prop}     [thm]{Proposition}
\newtheorem{property} [thm]{Property}
\newtheorem{defin}    [thm]{Definition}
\newtheorem{corollary}     {Corollary}

\begin{document}
  \noindent COM 5165 Multiple Antenna Communication \hfill 113064501 Chun-Ting Lin\\

  \begin{center}
    {\bf \large  Homework III}
  \end{center}


  %--------------------------------------------------------------
  \begin{enumerate}
    \item[{\bf 2. }]  \textbf{$\epsilon$-outage probability} \hfill \\
      \vspace{-10pt}
\begin{itemize}
    \item[(a)] No. Recall that one Erlang $\leftrightarrow$ one hour of call traffic during one gour of 
    operatiom. Therefore, the total traffic load cannot be greater than the number of available channel.
    \item[(b)] However, there are some results from (a) that do not satisfy the requirement above i.e $\rho >
    m$. The reason is that the actual traffic is determined by traffic load $\rho$ and blocking probability 
    $B(\rho, m)$, that is
    \begin{equation*}
        \text{actual traffic} = \hat{\rho} = (1 - B(\rho, m)) \rho
    \end{equation*}
\end{itemize}
  \end{enumerate}
\end{document}
