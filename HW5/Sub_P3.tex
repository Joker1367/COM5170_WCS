Similar to that in problem 2, if the incidence angle of the desired signal is $\phi_i$, then the
interference power is obtained by projecting other signals onto the steering vector of desired signal.
\begin{equation*}
    \text{P}{\scriptsize \text{interference}} = P \cdot \sum_{j \neq i} \left|e_r^{H}(\Omega_j) \cdot e_r(\Omega_i)\right|^2.
\end{equation*}
Hence, the signal-to-interference power ratio (SIR) is given by
\begin{equation*}
    \text{SIR} = \frac{\left|e_r^{H}(\Omega_i) \cdot e_r(\Omega_i)\right|^2}{\sum_{j \neq i} \left|e_r^{H}(\Omega_j) \cdot e_r(\Omega_i)\right|^2}.
\end{equation*}
The SIRs of each signal are listed in the table below:
\begin{table}[H]
    \centering
    \begin{tabular}{c|c|c|c|c|c}
             & Signal 1 & Signal 2 & Signal 3 & Signal 4 & Signal 5 \\
    \hline
    SIR (dB) & 17.7617 & 1.0028 & 13.3221 & 17.8966 & 1.2187
    \end{tabular}
\end{table}
From the table above, it can be observed that the SIR of signals 2 and 5 is close to one.
The reason is that $\phi_2$ and $\phi_5$ are close to each other. The steering vector of signal 2
also provides gain for signal 5, and vice versa. More precisely,
\begin{equation*}
    \left|e_r^{H}(\Omega_2) \cdot e_r(\Omega_2)\right|^2 \approx \left|e_r^{H}(\Omega_5) \cdot e_r(\Omega_2)\right|^2.
\end{equation*}
and
\begin{equation*}
    \left|e_r^{H}(\Omega_5) \cdot e_r(\Omega_5)\right|^2 \approx \left|e_r^{H}(\Omega_2) \cdot e_r(\Omega_5)\right|^2.
\end{equation*}