
\begin{itemize}
    \item \textbf{Filtered Gaussian Noise method} \hfill \\
    In filtered gaussian noise method, we may notice that the \textcolor{blue}{fluctuation of the 
    envelope increases as $f_mT$ getting bigger.} On the other hand, the auto correlation is not affected 
    by $f_mT$ a lot.
    \begin{itemize}
        \item Advantage: Different paths are \textcolor{blue}{uncorrelated.}
        \item Disadvantage: This method is based on first-order filter. Therefore, the power 
        spectrum density of the generated siganl is much \textcolor{blue}{different from the ideal case 
        (U shape).} To have a more accurate result, we can use a higher order filter, which increases the complexity.
    \end{itemize}
    \item \textbf{Sum of Sinusoids method} \hfill \\
    In sum of sinusoids method, we may notice that \textcolor{blue}{the fluctuation of the envelope 
    increases as $f_mT$ getting bigger.} Moreover, \textcolor{blue}{autocorrelation behaves more like the ideal 
    autocorrelation as $M$ increases.} That is, for larger $M$, the undesired behavior occurs 
    later.
    \begin{itemize}
        \item Advanatges: generate \textcolor{blue}{isotropic scattering} fading environment with low complexity 
        by using less oscillator.
        \item Disadvantage: There's \textcolor{blue}{no randomness} in the generating process. Therefore, in order
        to use it for modeling the real case, some modification is needed. For example, different 
        simulation should start from different point to confront different fading.
    \end{itemize}
\end{itemize}